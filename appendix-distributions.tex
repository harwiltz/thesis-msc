\chapter{Tempered Distributions}
A recurring concept in many areas of mathematics, physics, and engineering is
that of \emph{generalized functions}, known as \emph{distributions}\footnote{Not
  to be confused with probability distributions.}. One such example is the Dirac
  delta. Distributions are particularly helpful at formally describing weakened solutions
  to PDEs by objects that may not be functions.

In this thesis, we will make use of the class of \emph{tempered} distributions,
whose definition will be given in this appendix. For more details, refer to
\citet{lax2002functional}.

\begin{definition}[Schwartz Class]
  Let $X$ be a normed space. A \emph{Schwartz class} is a class $\mathcal{S}$ of rapidly decaying-smooth
  functions,

  \begin{align*}
    \mathcal{S} = \left\{f\in C^\infty(X; \mathbf{R}) : \sup_{x\in X}(1 +
    \|x\|^k)|f^{(m)}(x)|<\infty\quad\forall k,m\in\mathbf{N}\right\}
  \end{align*}
\end{definition}

\begin{definition}[Tempered Distribution]\label{def:tempered-distribution}
  A tempered distribution is an element of the topological dual\footnote{The
    dual of a normed space is the set of all continuous, linear functionals on
    that space.} $\mathcal{S}'$
  of the Schwartz class $\mathcal{S}$.
\end{definition}

\begin{remark}
  The Dirac delta is the operator $\delta$ such that $\langle\delta, \phi\rangle
  = \phi(0)$. Clearly $\delta$ is linear, and since it is bounded,
  it is continuous. Therefore $\delta$ is indeed a tempered distribution.
\end{remark}

Tempered distributions admit a notion of differentiability, which can be used to
define ``distributional" solutions to PDEs.

\begin{definition}[Distributional
  Derivative]\label{def:distributional-derivative}
  Let $\mathcal{S}$ be a Schwartz class and $\psi\in\mathcal{S}'$ a tempered
  distribution. Then $\psi$ has a distributional derivative if there exists a
  tempered distribution $\psi'$ for which

  \begin{align*}
    \langle \psi', \phi\rangle &= -\langle\psi,
    \phi'\rangle\qquad\forall\phi\in\mathcal{S},
  \end{align*}

  and $\psi'$ is called the distributional derivative of $\psi$.
\end{definition}

\begin{definition}[Distributional Solutions of Hamilton-Jacobi
  PDEs]\label{def:distributional-solution}
  Consider the following PDE,
  \begin{equation}\label{eq:distributional-solution:pde}
    \partialderiv{u}{t} = f\circ u + \langle\nabla u, g\rangle +
    \quadraticform{h}{\hessian{y}u}
  \end{equation}

  where $u\in C^2(\mathbf{R}_+\times\mathcal{Y};\mathbf{R})$ for a normed space
  $\mathcal{Y}$.

  Then $\psi\in\mathcal{S}'$ is said to be a \emph{distributional solution} to
  \eqref{eq:distributional-solution:pde} if

  \begin{align*}
    &\int_0^\infty\int_{\mathcal{Y}}\phi(t, y)\left(f(\psi(y)) -
      \partialderiv{}{t}\psi(y)\right)dydt\\
    &\qquad=\int_{0}^\infty\int_{\mathcal{Y}}\bigg[\langle \psi(y)g(y), \nabla_y\phi(t,
    y)\rangle - \quadraticform{h(y)}{\psi(y)\hessian{y}\phi(t,
    y)}\bigg]dydt
  \end{align*}

  for every test function $\phi\in\mathcal{S}$. This is justified by simply multiplying both sides of
  \eqref{eq:distributional-solution:pde} by the test function, integrating over
  $\mathbf{R}_+\times\mathcal{Y}$, and substituting gradient terms of $\psi$
  with respect to its distributional derivative.
\end{definition}
