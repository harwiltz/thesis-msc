\chapter{A Primer on Topology}\label{app:analysis}
Concepts from the field of topology are mentioned often in this
thesis. Some definitions and basic results are stated here.

\begin{definition}[Topological Space]\label{def:topology}
  A \textbf{topological space} is a pair $(X, \mathcal{O})$ consisting of a set
  $X$ and a collection $\mathcal{O}\subset 2^X$ of subsets of $X$ such
  that
  \begin{enumerate}
  \item $\emptyset\in\mathcal{O}$ and $X\in\mathcal{O}$;
  \item If $\indexedin{I}{\alpha}{U}\subset\mathcal{O}$, then
    $\bigcup_{\alpha\in I}U_\alpha\in\mathcal{O}$;
  \item If $N$ is a finite integer and
    $\indexedint{i}{N}{U}\subset\mathcal{O}$,
    then $\bigcap_{i\in[N]}U_i\in\mathcal{O}$.
  \end{enumerate}

  The set $\mathcal{O}$ is called a \textbf{topology} on $X$, and the
  elements of $\mathcal{O}$ are called \textbf{open sets}.
\end{definition}

It is only natural to ask what it means for a set to be closed.

\begin{definition}[Closed Set]
  Let $(X,\mathcal{O})$ be a topological space. A set $F\subset X$ is
  said to be \textbf{closed} if its complement is an open set.
\end{definition}

\begin{remark}
  It should be noted that openness and closedness are not mutually
  exclusive properties of sets -- in fact, by the very definition of a
  topology, the ``whole space'' and the empty set must both be
  simultaneously open and closed. Such sets are called \textbf{clopen}.
\end{remark}

The choice of the topology characterizes what it means for a function
to be continuous and what it means for a sequence to converge (among
other things).

\begin{definition}[Continuous Function]\label{def:continuity}
  Let $(X,\mathcal{O}), (Y,\mathcal{U})$ be topological spaces. A
  function $f:(X, \mathcal{O})\to(Y, \mathcal{U})$ is said to be
  continuous if its preimage of every open set $U\subset Y$ is an open
  set in $X$. That is,

  \begin{align*}
    U\in\mathcal{U}\implies\{x\in X: f(x)\in U\}\in\mathcal{O}
  \end{align*}
\end{definition}

\begin{definition}[Convergence]\label{def:convergence}
  Let $(X,\mathcal{O})$ be a topological space. A sequence
  $\indexedint{i}{N}{x}\subset X$ is said to \textbf{converge} to a
  point $x\in X$ if for every open set $U\ni x$ there exists a finite
  integer $N$ such that $\indexedint[N]{i}{\infty}{x}\subset U$.
\end{definition}

The following proposition can be verified directly.

\begin{proposition}[The Universal Topology]
  \label{pro:universal-topology}
  Let $X = \mathbf{N}$ and let
  \begin{align*}
    \mathcal{O} = \{\emptyset, X, U, X\setminus U\}
  \end{align*}
  where $U = \{4,8,15,16,23,42\}$. Then any sequence
  $\indexedint{i}{\infty}{x}\subset X$ such that
  $\indexedint[N]{i}{\infty}{x}\subset U$ converges to $42$, where $N$
  is a finite integer. For instance, the sequence $15, 16, 15, 16,
  \dots\to 42$.
\end{proposition}

\section{Metric Spaces}\label{s:metric-space}
Proposition \ref{pro:universal-topology} should be a little
alarming. Indeed, many topological spaces are quite
pathological. Usually we restrict our interests to spaces with a
little more structure, such as a spaces that can be equipped with a
meaningful notion of distance.

\begin{definition}[Metric Space]\label{def:metric-space}
  A \textbf{metric space} is a pair $(X, d_X)$ where $X$ is a set and
  $d_X:X\times X\to\mathbf{R}_+$, called a \textbf{metric} or a
  distance function, satisfies
  \begin{enumerate}
  \item (\emph{Separation of points}) For any $x,y\in X$, $d_X(x, y) = 0\iff
    x = y$;
  \item (\emph{Symmetry}) For any $x,y\in X$, $d_X(x, y) = d_X(y, x)$;
  \item (\emph{Triangle inequality}) For any $x,y,z\in X$, $d_X(x,
    z)\leq d_X(x, y) + d(y, x)$.
  \end{enumerate}
\end{definition}

A metric space is a special case of a topological space, where the
topology is understood to be the smallest topology\footnote{The
  \emph{smallest} topology conforming to some constraint is the
  intersection of all topologies that conform to the constraint.} containing all open
balls $B_r(x) = \{y\in X : d_X(x, y)<r\}$. Notably, not every
topological space has a metric structure. For instance, there is no
function on $\mathbf{N}$ with the
\hyperref[pro:universal-topology]{universal topology} that satisfies
the metric properties.

\begin{remark}
  The definitions of \hyperref[def:continuity]{continuity} and
  \hyperref[def:convergence]{convergence} on metric spaces coincide
  with those that are given in standard calculus courses.
\end{remark}

\begin{definition}[Cauchy Sequence]
  Let $(X, d)$ be a metric space. A sequence
  $\indexedint{i}{\infty}{x}\subset X$ is called a \textbf{Cauchy
    sequence} if for every $\epsilon>0$ there exists a finite integer
  $N$ such that

  \begin{align*}
    d(x_n, x_m)\leq\epsilon\qquad\forall m,n\geq N
  \end{align*}
\end{definition}

\begin{remark}
  Counterintuitively, Cauchy sequences may not converge. A sequence
  only converges (in a given topological space) if the limit lies in
  the space. For instance, the sequence
  $\indexedint{k}{\infty}{x}\subset (0,1)$ where $x_k=\frac{1}{k}$ is
  Cauchy, but its limiting value is $0$ which is not in $(0,1)$.
\end{remark}

\begin{definition}[Complete Space]\label{def:complete}
  A metric space $(X, d)$ is said to be \textbf{complete} if every
  Cauchy sequence in $X$ converges in $X$. By the previous remark, the
  set $(0,1)$ is not complete with the standard topology on the real numbers.
\end{definition}

Finally, we demonstrate a useful property of metric spaces.

\begin{lemma}[Well-behaved convergence]\label{lem:metric:convergence}
  In any metric space $(X, d)$, no sequence can converge to more than
  one point.
\end{lemma}
\begin{proof}
  Suppose $\indexedint{k}{\infty}{x}\subset X$ has limits $x, y$. Then

  \begin{align*}
    \lim_{k\to\infty}d(x_k, x) &= 0\\
  \end{align*}

  Moreover, by the triangle inequality, for any $k$ we have $d(x,
  y)\leq d(x, x_k) + d(x_k, y)$. Therefore,

  \begin{align*}
    d(x, y) &\leq \lim_{k\to\infty}d(x, x_k) + \lim_{k\to\infty}d(x_k,
              y)\\
    &= d(x_k, y)
  \end{align*}
  So if $x_k\to y$, then we must have $d(x, y) = 0$, and by the
  separation of points property this implies that $x=y$.
\end{proof}
