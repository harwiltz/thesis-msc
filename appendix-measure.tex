\chapter{The Basics of Measure Theory}\label{app:measure}

\newcommand{\vol}{\mathsf{Vol}}

Measure theory is a vast field of mathematical analysis that is
concerned with generalizing the notion of \emph{measure}, such as
length, area, and volume, to arbitrary spaces. For the sake of building
intuition, suppose we have a 3-dimensional sphere $S =
\{x\in\mathbf{R}^3 : \|x\|\leq r\}$ and we are interested in measuring
its volume, as well as the volume of arbitrary ``pieces'' of the
sphere. Formally, we're looking for a function $\mathsf{Vol}:2^S\to
\mathbf{R}_+$ that maps subsets of the sphere to a non-negative real
number. This function cannot just be an arbitrary function, as we
expect a volume to satisfy certain properties. For instance, we need
the following:

\begin{enumerate}
\item Emptiness has no volume: $\vol(\emptyset) = 0$;
\item For disjoint subsets $A,B\subset S$, the volume of the
  combination of $A,B$ should be equivalent to the sum of their
  original volumes: $A\cap B=\emptyset\implies\vol(A + B) = \vol(A) +
  \vol(B)$;
\item For any subset $D\subset S$, no subset of $D$ can have more
  volume than $D$: $C\subset D\implies\vol(C)\leq\vol(D)$.
\end{enumerate}

This is not particularly interesting at first glance. However, with
this definition of measure, there are some alarming consequences. In
particular, the \emph{Banach-Tarski paradox} demonstrates how one can
disassemble $S$ into a collection of pieces, and reassemble the pieces
to form two identical copies of $S$
\citep{banach1924decomposition}. Moreover, a function that
measures the length of arbitrary subsets of the real
line (according to the rules above) and assigns finite length to the interval $(0,1)$ \emph{cannot
  possibly exist} \citep{cohn2013measure}.

Interestingly, the issue lies not exactly with the rules we listed,
but with the sets we wish to measure. In short, in order to construct
a meaningful measure function, we must restrict this function to
measure only ``nice-enough'' sets. In this context, the collection of ``nice
enough'' sets is actually quite vast, and it is usually difficult to
even conceive a set that is not nice enough. This will be explained in
further detail in \S\ref{app:measure:measure}, and integration with
respect to measures will be discussed in
\S\ref{app:measure:integration}.

\section{Measurable Spaces}\label{app:measure:measure}

The distinction of sets that can and cannot be measured is formalized
by a \emph{measurable space}. Not unlike topological spaces, a
measurable space is comprised of a set with a collection of subsets,
where the subsets denote the sets that one may measure. Like the
topology of a set, the collection of measurable sets cannot be
arbitrary. Rather, it must be a \emph{$\sigma$-algebra}.

\begin{definition}[$\sigma$-algebra]\label{def:sigma-algebra}
  Let $\Omega$ be a set. A \emph{$\sigma$-algebra over $\Omega$} is a
  collection of subsets $\Sigma\subset 2^\Omega$ such that

  \begin{enumerate}
  \item $\emptyset, \Omega\in\Sigma$;
  \item $A\in\Sigma\implies A^c = \Omega\setminus A\in\Sigma$;
  \item If $\indexedint{k}{\infty}{A}$ is a countable collection of sets in
    $\Sigma$, then $\bigcup_kA_k\in\Sigma$.
  \end{enumerate}

  We occasionally refer to the \emph{smallest $\sigma$-algebra}
  containing a collection of subsets, or the $\sigma$-algebra
  \emph{generated by} this collection of subsets. For a collection of
  subsets $\mathcal{U}\subset 2^\Omega$, this $\sigma$-algebra is denoted by
  $\sigma(\mathcal{U})$ and is defined as

  \begin{align*}
    \sigma(\mathcal{U}) = \bigcap\left\{\Sigma\in 2^\Omega :
    \Sigma\text{ is a $\sigma$-algebra},\quad \mathcal{U}\subset\Sigma\right\}
  \end{align*}
\end{definition}

Essentially, the $\sigma$ algebra describes any quantities we may want
to measure. If we are able to measure the volume of $S$ and we are
able to measure the volume of $A\subset S$, then naturally we should
be able to measure the volume of $S\setminus A$. Likewise, if we can
measure the volume of a countable collection of subsets of $S$, we
should be able to measure their union. While this construction seems
rather innocuous, $\sigma$-algebras can contain exceptionally
``rough'' sets. Below, we define a family of $\sigma$-algebras that is
referred to extensively in this thesis.

\begin{definition}[Borel $\sigma$-algebra]\label{def:borel}
  Let $(X, \mathcal{O})$ be a \hyperref[def:topology]{topological
    space}. The \emph{Borel $\sigma$-algebra over $(X,\mathcal{O})$}
  (or the Borel $\sigma$-algebra over $X$ when the topology is
  implicit), denoted by $\mathscr{B}(X, \mathcal{O})$, is the
  smallest $\sigma$-algebra containing $\mathcal{O}$.
\end{definition}

\begin{definition}[Measurable Space]\label{def:measurable-space}
  A \emph{measurable space} is a pair $(\Omega, \Sigma)$ where
  $\Omega$ is a set and $\Sigma$ is a $\sigma$-algebra over $\Omega$.
\end{definition}

We are finally able to formalize the concept of a measure.

\begin{definition}[Measure]\label{def:measure}
  Let $(\Omega, \Sigma)$ be a measurable space. A \emph{measure} on
  $(\Omega,\Sigma)$ is a function $\mu:\Sigma\to\mathbf{R}_+$ such
  that

  \begin{enumerate}
  \item $\mu(\emptyset) = 0$;
  \item $A\subset B\implies \mu(A)\leq\mu(B)$;
  \item If $\indexedint{k}{\infty}{A}$ is a countable collection of disjoint
    sets in $\Sigma$ (so $i\neq j\implies A_i\cap A_j=\emptyset$),
    then
    \begin{align*}
      \mu\left(\bigcup_{k=1}^\infty A_k\right) = \sum_{k=1}^\infty \mu(A_k)
    \end{align*}
  \end{enumerate}

  A tuple $(\Omega, \Sigma, \mu)$ is called a \emph{measure space}.
\end{definition}

Taking a step back to the examples above, it is known that there is no
measure on the measurable space $(\mathbf{R}, 2^{\mathbf{R}})$ that
assigns finite measure to $(0,1)$, and matter can be created out of
thin air if we can break apart objects into any subset of space.

A very important result in measure theory is the existence of a measure
space over $\mathbf{R}$ that assigns the measure $|b - a|$ to subsets
of the form $(a, b), [a, b], (a, b], [a, b)$. This measure is called
\emph{the Lebesgue measure}\label{def:lebesgue}, and it is the only
measure satisfying the mentioned property. See
\citet{cohn2013measure}, or any textbook on measure theory, for more
rigorous details.

Finally, we'll define the class of functions that preserve
measure-theoretic properties.

\begin{definition}[Measurability]
  Let $(\Omega, \Sigma), (\Omega', \Sigma')$ be measurable spaces. A
  function $f:(\Omega, \Sigma)\to(\Omega', \Sigma')$ is said to be
  \emph{measurable} if the preimage of every $\Sigma'$-measurable set
  $A$ through $f$ is $\Sigma$-measurable.
\end{definition}

It can quickly be verified that the composition of measurable
functions is itself a measurable function
\citep{cohn2013measure}. Therefore, we can define measures through a
change of variables, assuming the mapping between variables is
measurable.

\begin{definition}[Pushforward Measure]
  Let $(\Omega, \Sigma, \mu)$ be a measure space, and let $(\Omega',
  \Sigma')$ be a measurable space. For any measurable function
  $f:(\Omega, \Sigma)\to(\Omega',\Sigma')$, the \emph{pushforward of $f$
    through $\mu$}, denoted $\pushforward{f}{\mu}$, is a measure on
  $(\Omega', \Sigma')$ given by

  \begin{align*}
    \pushforward{f}{\mu} = \mu\circ f^{-1}
  \end{align*}

  where $f^{-1}$ is the preimage of $f$.
\end{definition}

\subsection{Measure-theoretic Probability Theory}\label{s:app:measure:probability}
A natural application of this formalism, aside from measurements of
geometric properties, is probability. In fact, we can formalize
probability very easily as a measure space.

\begin{definition}[Probability Space]\label{def:probability}
  A \emph{probability space} is a measure space $(\Omega, \Sigma,
  \mu)$ where $\mu(\Omega) = 1$.
\end{definition}

Occasionally, in the context of probability, the set $\Omega$ is
called the \emph{sample space}, the $\sigma$-algebra $\Sigma$ is
called the \emph{event space}, and the measure $\mu$ is called a
\emph{probability measure}.

Moreover, we can use the language of measure theory to formalize the
concept of a random variables.

\begin{definition}[Random Variable]\label{def:random-variable}
  Let $(\Omega, \Sigma, \mu)$ be a probability space, and let $A$ be
  an arbitrary set. A \emph{random variable} on this space is a
  function $Y:\Sigma\to A$.
\end{definition}

For a given measure space $(\Omega, \Sigma, \mu)$, a property is said
to hold \emph{$\mu$-almost everywhere} (or simply ``almost
everywhere'' when the measure is implicit) if the property holds on
all of $\Omega$, except for possibly a set $A$ with $\mu(A) = 0$. When
$\mu$ is a probability measure, it is sometimes said that the property
holds \emph{almost surely}.

\section{Integration}\label{app:measure:integration}
A measure can be thought of as an arbitrary method of assigning weight
or density to a space. As such, the notation of integration can be
formulated in terms of measures. In this section, a brief overview of
this type of integration, known as Lebesgue integration, and its properties will be given.

We'll consider a measure space $(\Omega, \Sigma, \mu)$. In order to
construct an integral, we'll begin by defining the integral on a
simple class of functions, aptly called the \emph{simple functions}.

\begin{definition}[Simple Function]\label{def:simple-function}
  A \emph{simple function} $f$ is a function of the form

  \begin{align*}
    f(x) &= \sum_{i=1}^n\alpha_i\chi_{A_i}(x)
  \end{align*}

  where $\alpha_i\in\mathbf{R}$ and $\indexedint{i}{n}{A}$ is a finite
  collection of measurable sets.
\end{definition}

It is easy to verify that the sum and product of simple functions are
both simple functions. The notion of integration of a simple function
$f$ with respect to a measure is fairly intuitive. We define

\begin{align*}
  \int f(x)d\mu &= \sum_{i=1}^n\alpha_i\mu(A_i)\qquad f = \sum_{i=1}^n\alpha_i\chi_{A_i}
\end{align*}

By linearity, it clearly follows that the Lebesgue integral restricted
to simple functions is linear. The Lebesgue integral of measurable
functions $f$ is given by

\begin{align*}
  \int fd\mu &= \sup\left\{\int sd\mu : s\text{ is a simple function}\right\}
\end{align*}

It is well known that the Lebesgue integral is linear over all
measurable functions, and it is well defined. Moreover, it is known
that the Lebesgue integral with respect to the Lebesgue measure agrees
with the Riemann-Stieltjes integral on all integrable functions.

\subsection{Convergence
  Theorems}\label{s:app:measure:integration:convergence}
In this section, we'll simply state some commonly known convergence
properties of the Lebesgue integral. See \citet{cohn2013measure} for
further details.

\begin{theorem}[The Monotone Convergence Theorem]\label{thm:monotone-convergence}
  Let $(\Omega, \Sigma, \mu)$ be a measure space and let
  $\indexedint{i}{\infty}{f}$ be a sequence of $[0,\infty]$-valued
  $\Sigma$-measurable functions. Suppose that for all $f_i\leq f_j$
  for all $i\leq j$, and that $\lim_{i\to\infty}f_i(x) = f(x)$ for
  almost every $x\in X$. Then $\int fd\mu = \lim_{i\to\infty}\int f_id\mu$.
\end{theorem}

\begin{theorem}[The Dominated Convergence
  Theorem]\label{thm:dominated-convergence}
  Let $(\Omega, \Sigma, \mu)$ be a measure space, $g$ a
  $[0,\infty]$-valued integrable function on $\Omega$, and
  $\indexedint{n}{\infty}{f}$ a collection of $\Sigma$-measurable
  functions where $\lim_{n\to\infty}f_n(x) = f(x)$ almost
  everywhere. If $|f_n(x)|\leq g(x)$ almost everywhere for each $n$,
  then $\indexedint{n}{\infty}{f}$ and $f$ are integrable, and $\int f(x) =
  \lim_{n\to\infty}\int f_nd\mu$ almost everywhere.
\end{theorem}
